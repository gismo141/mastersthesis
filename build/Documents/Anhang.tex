\chapter{Anhang}\label{anhang}

\section{Reproduktion der Forschung}\label{reproduktion-der-forschung}

Zur Reproduktion der Forschung wird im Folgenden die
Entwicklungsumgebung vorgestellt. Für konkrete Information zur
Installation wird einerseits auf die jeweilige Dokumentation für das
Programm sowie auf das Github-Repository
\href{https://github.com/gismo141/laserIMUCalibration}{gismo141/laserIMUCalibration}
als Begleitmaterial der vorliegenden Arbeit verwiesen.

Die Implementierung der Algorithmen erfolgte einerseits in Python
(Fast-Implementation) und C++ zur Integration in das vorhandene
Framework. Dabei wurden die Betriebssysteme Mac OS X 10.10.3 und Windows
7 verwendet. Es wurde darauf Wert gelegt, Open-Source Software zu
verwenden.

\subsection{Entwicklungsumgebung}\label{entwicklungsumgebung}

Zur bequemen Installation der benötigten Programme wird für Mac OS X ein
Paketmanager wie Homebrew oder MacPorts empfohlen. Eventuelle
Abhängigkeiten können somit während der Installation festgestellt und
aufgelöst werden. Homebrew trennt dabei die vom Benutzer installierten
Programme von den system-eigenen wodurch Komplikationen vermindert
werden.

Homebrew stellt keine Notwendigkeit dar, die benötigte Software kann
auch von Hand heruntergeladen, kompiliert und installiert werden.

\begin{longtable}[c]{@{}lll@{}}
\caption{Notwendige Programme zur Erstellung der Arbeit und Ihrer
Abhängigkeiten}\tabularnewline
\toprule
Tool & Version & Verwendung\tabularnewline
\midrule
\endfirsthead
\toprule
Tool & Version & Verwendung\tabularnewline
\midrule
\endhead
Qt5 & 5.4.1 & UI-Design (Oberflächenentwicklung)\tabularnewline
VTK & 6.2.0 & Visualisierung der Daten und Algorithmen\tabularnewline
PCL & 1.7.2 & Algorithmen zum Umgang mit Punktwolken (ICP, Outlier
etc.)\tabularnewline
CMake & 3.2.2 & einheitliche Build-Umgebung\tabularnewline
Python & 2.7 & schnelles Algorithmen-Prototyping\tabularnewline
Pandoc & 1.13.2.1 & Erstellung der Masterarbeit in unterschiedlichen
Formaten\tabularnewline
GraphViz & 2.38.0 & Visualisierung der Graphen und
Algorithmen\tabularnewline
Doxygen & 1.8.9.1 & Dokumentation des Quellcodes\tabularnewline
Gnuplot & 5.0.0 & Plotten der gewonnenen Daten\tabularnewline
\bottomrule
\end{longtable}

\subsection{Interessante Links}\label{interessante-links}

\begin{itemize}
\itemsep1pt\parskip0pt\parsep0pt
\item
  \href{http://rawgit.com}{RawGit (Extrahiert Links aus
  Github-Repositories)}
\item
  \href{http://stackoverflow.com/a/11939703/3281871}{PCLVisualizer in Qt
  verwenden}
\item
  \href{https://github.com/strawlab/python-pcl}{Python-PCL}
\item
  \href{pointclouds.org/documentation/tutorials/adding_custom_ptype.php}{Erstellung
  eigener Punkt-Datentypen in der PCL}
\end{itemize}
